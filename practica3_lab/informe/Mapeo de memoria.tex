\documentclass{article}
\usepackage[utf8]{inputenc}
\usepackage{amsmath} % For mathematical environments if needed
\usepackage[margin=1in]{geometry} % Add this line to set margins

\title{Práctica 3: Mapeo de memoria E/S}
\author{Rolando Suarez V-30445947 \\
		Jesus Muñoz V-27188137}
\date{}

\begin{document}
	
	\maketitle
	
	
	\section{Organización de la memoria con Memory-Mapped I/O}
	
	Cuando un sistema utiliza \textbf{memory-mapped I/O (MMIO)}, los registros de los dispositivos de entrada/salida (E/S) se asignan a direcciones dentro del espacio de direcciones de la memoria principal. Esto significa que el procesador interactúa con los dispositivos de E/S como si estuviera leyendo o escribiendo en ubicaciones de memoria normales.
	
	En la arquitectura \textbf{MIPS32}, la memoria principal se organiza en varias secciones:
	\begin{itemize}
		\item \textbf{Segmento para texto}: Comienza en la dirección \texttt{0x00400000}.
		\item \textbf{Segmento de datos}: Comienza en la dirección \texttt{0x10010000}.
		\item \textbf{Heap}: Memoria dinámica que crece hacia arriba para almacenamiento de datos.
		\item \textbf{Stack}: Memoria para registros y variables de ejecución, crece hacia abajo.
		\item \textbf{Segmento de texto del kernel}: Comienza en \texttt{0x80000000}.
		\item \textbf{Segmento de datos del kernel}: Comienza en \texttt{0x90000000}.
	\end{itemize}
	
	Los registros de \textbf{Memory-Mapped I/O} para los dispositivos de E/S se mapean típicamente en una región de memoria específica. En el simulador MARS, esta región comienza físicamente en la dirección \textbf{\texttt{0xFFFF0000}}. Esta es la región donde se mapean los dispositivos.
	
	Las instrucciones \texttt{lw} (load word) y \texttt{sw} (store word) se utilizan como funciones de entrada y salida para comunicarse con los dispositivos, con la memoria como intermediario. La instrucción \texttt{lw} sirve como función de lectura; por ejemplo, para leer si una tecla del teclado ha sido presionada. La instrucción \texttt{sw} sirve para lo contrario, es decir, para escribir datos en un dispositivo, como enviar un carácter a una pantalla.
	
	
	\section{Diferencia entre Memory-Mapped I/O y E/S por puertos}
	
	La principal diferencia entre \textbf{Memory-Mapped I/O (MMIO)} y la \textbf{entrada/salida por puertos (PMIO)} radica en el espacio de direcciones utilizado para los dispositivos de E/S.
	
	\subsection{Ventajas y Desventajas de MMIO}
	\textbf{Ventajas de MMIO:}
	\begin{itemize}
		\item \textbf{No requiere instrucciones especiales}: Utiliza las mismas instrucciones de carga (\texttt{lw}) y almacenamiento (\texttt{sw}) para acceder a los datos de los dispositivos, lo que simplifica el conjunto de instrucciones del procesador.
		\item \textbf{Permite usar punteros y estructuras}: Facilita el acceso a los dispositivos utilizando las técnicas de programación de memoria estándar.
		\item \textbf{Facilita la integración con compiladores y lenguajes de alto nivel}: Al tratar los dispositivos como ubicaciones de memoria, es un método más flexible y portable.
	\end{itemize}
	\textbf{Desventajas de MMIO:}
	\begin{itemize}
		\item \textbf{Reduce el espacio disponible en RAM}: Parte del espacio de direcciones de memoria se designa para los dispositivos de E/S, lo que puede limitar la cantidad de RAM disponible para programas.
		\item \textbf{Puede requerir protección adicional}: Es necesario implementar mecanismos de protección para evitar accesos accidentales o no autorizados a los registros de los dispositivos.
	\end{itemize}
	
	\subsection{Ventajas y Desventajas de PMIO}
	\textbf{Ventajas de PMIO:}
	\begin{itemize}
		\item \textbf{No ocupa la memoria física}: Utiliza un espacio de direcciones separado y dedicado para la entrada y salida, lo que deja todo el espacio de direcciones de memoria principal disponible para la RAM.
		\item \textbf{Puede simplificar la protección de memoria}: Al tener un espacio de direcciones distinto, la protección de los dispositivos puede ser más directa.
	\end{itemize}
	\textbf{Desventajas de PMIO:}
	\begin{itemize}
		\item \textbf{Requiere instrucciones especiales}: Necesita instrucciones específicas como \texttt{IN} y \texttt{OUT} para interactuar con los puertos de E/S.
		\item \textbf{Menos intuitivo para programadores de alto nivel}: La interacción con los dispositivos puede ser menos directa ya que no se utiliza el modelo de memoria estándar.
	\end{itemize}
	
	\subsection{Razones del uso de MMIO en MIPS32}
	\textbf{MIPS32}, al igual que otras arquitecturas RISC, utiliza principalmente \textbf{memory-mapped I/O} por varias razones clave:
	\begin{itemize}
		\item \textbf{Simplicidad del conjunto de instrucciones}: Evita la necesidad de añadir instrucciones especiales para las tareas de entrada y salida, manteniendo el conjunto de instrucciones lo más pequeño y eficiente posible.
		\item \textbf{Facilidad de programación}: Permite usar las mismas instrucciones, herramientas y técnicas para el acceso a memoria y a los dispositivos, lo que simplifica el desarrollo de software.
		\item \textbf{Diseño limpio y uniforme}: Facilita la estandarización y uniformidad en el uso de dispositivos de entrada y salida, contribuyendo a un diseño de sistema más coherente.
	\end{itemize}
	
	\section{Problemas de direcciones solapadas en MMIO y cómo evitarlos}
	
	En un sistema con \textbf{memory-mapped I/O}, es crucial que cada dispositivo de entrada y salida posea una dirección de memoria única asociada. Si existe un \textbf{solapamiento} de direcciones entre dos o más dispositivos, surgen graves problemas de funcionamiento:
	\begin{itemize}
		\item \textbf{Ambigüedad del procesador}: El procesador no podría distinguir a qué dispositivo está accediendo, lo que llevaría a comportamientos impredecibles. Por ejemplo, podría intentar leer datos de una pantalla cuando espera la entrada de un teclado.
		\item \textbf{Inestabilidad y falta de fiabilidad}: El sistema en su conjunto se vuelve inestable y poco fiable.
		\item \textbf{Corrupción de datos}: Pueden ocurrir problemas de corrupción de datos si el sistema detecta inconsistencias, ya que los datos podrían provenir de dispositivos distintos.
	\end{itemize}
	
	Para evitar este conflicto, se deben emplear las siguientes estrategias:
	\begin{itemize}
		\item \textbf{Asignación de direcciones fija y documentada}: Cada tipo de dispositivo debe tener un rango de direcciones de memoria exclusivo, previamente definido y bien documentado.
		\item \textbf{Mapeo por hardware}: Componentes como los controladores de bus se encargan de asegurar que cada dirección se enrute al dispositivo correcto, gestionando el acceso de manera ordenada.
	\end{itemize}
	
	\section{Simplificación del diseño del conjunto de instrucciones con MMIO}
	
	El \textbf{memory-mapped I/O (MMIO)} simplifica el diseño del conjunto de instrucciones de un procesador porque \textbf{no se requiere añadir nuevas instrucciones} para la comunicación con los dispositivos de entrada y salida. Toda la comunicación se realiza mediante las instrucciones ya existentes para cargar (\texttt{lw}, \texttt{lb}, \texttt{lh}, etc.) y almacenar (\texttt{sw}, \texttt{sb}, \texttt{sh}, etc.) datos. Esto resulta en:
	\begin{itemize}
		\item \textbf{Uniformidad del diseño}: El procesador no necesita lógica adicional para decodificar y ejecutar instrucciones de E/S específicas.
		\item \textbf{Portabilidad}: Al utilizar instrucciones estándar, el código de E/S es más portable entre diferentes implementaciones de la misma arquitectura.
		\item \textbf{Compatibilidad con la programación de alto nivel}: Los lenguajes de alto nivel pueden acceder a los dispositivos de E/S utilizando punteros y estructuras, de la misma manera que acceden a la memoria RAM.
	\end{itemize}
	
	Si se utilizara \textbf{E/S por puertos}, serían necesarias \textbf{instrucciones adicionales} como \texttt{IN} y \texttt{OUT}. Estas instrucciones tendrían que incluir:
	\begin{itemize}
		\item La \textbf{identificación del dispositivo} o puerto al que se quiere acceder.
		\item El \textbf{comando} o tipo de operación (lectura o escritura) que se desea realizar.
	\end{itemize}
	Esto añade complejidad al conjunto de instrucciones y requiere una lógica adicional en el hardware del procesador para decodificar y ejecutar estas instrucciones especiales. Los sistemas CISC, como los procesadores x86 de Intel y AMD, a menudo emplean este tipo de esquema.
	
	\section{Acceso a dispositivos mapeados en memoria a nivel de bus}
	
	Cuando el procesador accede a una dirección de memoria que corresponde a un dispositivo mapeado en memoria, ocurre lo siguiente a nivel del bus de datos y direcciones:
	\begin{itemize}
		\item El procesador coloca la \textbf{dirección del registro del dispositivo} en el bus de direcciones.
		\item También coloca las \textbf{señales de control} apropiadas (lectura/escritura) en el bus de control.
		\item En el caso de una escritura, los \textbf{datos a escribir} se colocan en el bus de datos.
	\end{itemize}
	
	El hardware sabe que debe acceder a un periférico en lugar de a la RAM porque:
	\begin{itemize}
		\item \textbf{Decodificación de direcciones}: Los circuitos de hardware de decodificación de direcciones en el sistema están diseñados para reconocer que ciertas rangos de direcciones no corresponden a la RAM física, sino a los registros de los dispositivos de E/S.
		\item \textbf{Espacio reservado}: Como se mencionó, existe un espacio físico de la memoria reservado específicamente para el mapeo de dispositivos (e.g., \texttt{0xFFFF0000} en MARS). Cuando una dirección cae dentro de este rango, el controlador de bus o la lógica de interconexión del sistema enruta la solicitud al controlador del dispositivo correspondiente en lugar de a los chips de RAM.
		\item \textbf{Controladores de dispositivos}: Cada dispositivo de E/S tiene un controlador específico que está conectado al bus. Este controlador responde únicamente a las direcciones mapeadas a su dispositivo, interpretando los accesos de memoria como comandos o lecturas/escrituras de sus registros internos.
	\end{itemize}
	En esencia, el dato que se está "guardando en memoria" es el comando o la información dirigida al dispositivo.
	
	\section{Acceso de programas normales a dispositivos mapeados en memoria}
	
	Generalmente, \textbf{no es posible} que un programa normal (sin privilegios) acceda directamente a un dispositivo mapeado en memoria. Esto se debe a mecanismos de protección implementados por el sistema operativo y el hardware:
	\begin{itemize}
		\item \textbf{Niveles de privilegio}: En la presencia de un sistema operativo, existen diferentes niveles de privilegio. Típicamente, el \textbf{nivel kernel} (o modo privilegiado) es el único que tiene acceso directo al espacio de direcciones reservado para el mapeado de E/S. Los programas de usuario (nivel usuario o modo no privilegiado) no tienen permiso para acceder a estas direcciones.
		\item \textbf{Sistema de conversión/traducción de direcciones}: Las unidades de manejo de memoria (MMU) en el procesador, bajo el control del sistema operativo, gestionan la traducción de direcciones virtuales a físicas. Estas MMU pueden configurarse para denegar el acceso a ciertas regiones de memoria (como las de E/S mapeadas) si el acceso proviene de un programa sin los privilegios adecuados.
	\end{itemize}
	En simuladores como MARS, estas protecciones a nivel de sistema operativo no están presentes, permitiendo un acceso más directo para fines de aprendizaje. Sin embargo, en un sistema real, la seguridad y estabilidad dependen de estas barreras de protección.
	
	\section{Técnicas para evitar esperas activas innecesarias}
	
	Al interactuar con dispositivos de entrada y salida, el procesador a menudo necesita esperar a que un dispositivo complete una operación o tenga datos listos. La \textbf{espera activa} (o \textit{polling}) implica que el procesador consulta repetidamente el estado del dispositivo, lo cual consume recursos de manera innecesaria, ya que el procesador generalmente opera a una velocidad mucho mayor que el dispositivo.
	
	Para evitar estas esperas activas innecesarias, se pueden emplear las siguientes técnicas:
	\begin{itemize}
		\item \textbf{E/S dirigida por interrupciones}: Esta es la técnica más común y eficiente. En lugar de que el procesador consulte al dispositivo, el dispositivo genera una \textbf{interrupción} cuando necesita atención del procesador (e.g., una operación ha finalizado, hay datos disponibles, o ha ocurrido un error). El procesador, al recibir la interrupción, suspende su tarea actual y ejecuta una rutina de servicio de interrupción (ISR) para atender al dispositivo. Esto permite que el procesador realice otras tareas mientras el dispositivo está ocupado, y solo se interrumpe cuando es estrictamente necesario.
		\item \textbf{Acceso Directo a Memoria (DMA)}: Para transferencias de datos grandes, la DMA permite que los dispositivos de E/S transfieran datos directamente hacia y desde la memoria principal sin la intervención constante del procesador. El procesador solo inicia la transferencia y es notificado mediante una interrupción cuando la transferencia DMA ha finalizado. Esto libera al procesador de la tarea de mover datos byte a byte o palabra a palabra.
	\end{itemize}
	
	\section{Análisis y Discusión de los Resultados}
	
	En la implementación de los algoritmos de sensor de temperatura y sensor de tensión, pudimos observar el funcionamiento del \textbf{mapeado de memoria para el control de dispositivos de entrada y salida}, al menos de forma simulada. Se creó un apartado de memoria con los nombres de los "comandos" que se deseaban asignar al dispositivo. Esto incluyó:
	\begin{itemize}
		\item Un \textbf{comando de control} para encender el dispositivo o prepararlo para la lectura.
		\item \textbf{Comandos de lectura} para obtener el valor de retorno que se supone que el dispositivo debería entregar.
	\end{itemize}
	En nuestro caso, estos datos se generaron de manera aleatoria, ya que no se contó con un medidor de tensión o termómetro real conectado al equipo.
	
	Esta práctica también demostró la utilidad de los scripts en MIPS32 destinados a generar números aleatorios dentro de un rango.
	
	\textbf{Nuestro algoritmo de temperatura} fue capaz de:
	\begin{itemize}
		\item Determinar si la lectura de temperatura fue correcta o si hubo un error.
		\item Mostrar en pantalla la temperatura leída.
	\end{itemize}
	
	\textbf{Nuestro algoritmo de tensión} mostró por pantalla:
	\begin{itemize}
		\item Un mensaje indicando que la medición se estaba realizando en ese momento.
		\item Los resultados de la medición de tensión sistólica y diastólica.
	\end{itemize}
	Estos resultados, aunque simulados, ilustran eficazmente cómo \textbf{memory-mapped I/O} permite la interacción del software con el hardware a través de operaciones de memoria estándar.
	
\end{document}