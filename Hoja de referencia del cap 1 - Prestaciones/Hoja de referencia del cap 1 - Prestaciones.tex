\documentclass{article}
\usepackage[utf8]{inputenc}
\usepackage[spanish]{babel}
\usepackage{amsmath}
\usepackage{amssymb}
\usepackage{booktabs}
\usepackage{longtable}
\usepackage{enumitem}
\usepackage{hyperref}
\usepackage{geometry}
\usepackage{tabularx} % Añade este paquete para tablas con ancho fijo
\usepackage{ragged2e} % Para justificar el texto en las celdas p{}

% Configuración de márgenes
\geometry{a4paper, margin=1in}
\title{Hoja de Referencia\\del Capítulo 1 – Prestaciones}
\author{Rolando Suarez\\
		V-30445947}
\date{}

\begin{document}
	
	\maketitle
	
	\section*{Prestaciones}
	Las prestaciones de un computador se refieren a la capacidad o eficiencia con la que una computadora realiza sus tareas y operaciones. En esencia, es una medida de qué tan bien y rápido puede un computador ejecutar programas, procesar datos y responder a las demandas del usuario.
	
	\subsection*{Conceptos Claves}
	\begin{enumerate}[label=\arabic*)]
		\item \textbf{Velocidad}
		\begin{enumerate}[label=\alph*., noitemsep]
			\item \textbf{Tiempo de respuesta (tiempo de ejecución):} Tiempo total requerido por un computador para completar una tarea, incluidos los accesos a discos, los accesos a memoria, las operaciones, sobrecarga del sistema operativo, tiempo de ejecución, etc.
			\item \textbf{Productividad (ancho de banda):} Se define como el número de tareas que se completan por unidad de tiempo.
		\end{enumerate}
	\end{enumerate}
	
	\subsection*{Cálculo de Prestaciones}
	Para maximizar las prestaciones, lo que se desea es minimizar el tiempo de respuesta o tiempo de ejecución del programa.
	
	Si las prestaciones de una máquina X son mayores que las de una máquina Y:
	\[ \text{Prestaciones}_X > \text{Prestaciones}_Y \]
	
	A menudo se desea relacionar cuantitativamente las prestaciones de dos máquinas diferentes. Con esto se podría decir que ``X es n veces más rápida que Y'':
	\[ \text{Prestaciones}_X = n \times \text{Prestaciones}_Y \]
	Esto implica que el tiempo de ejecución de Y es n veces mayor que el de X:
	\[ \text{Tiempo\_ejecución}_Y = n \times \text{Tiempo\_ejecución}_X \]
	De estas relaciones, se deduce que:
	\[ n = \frac{\text{Prestaciones}_X}{\text{Prestaciones}_Y} = \frac{\text{Tiempo\_ejecución}_Y}{\text{Tiempo\_ejecución}_X} \]
	
	\subsection*{Medición de las Prestaciones}
	\begin{itemize}
		\item Tiempo de ejecución – Segundos (s)
	\end{itemize}
	
	\subsection*{Conceptos Claves (II)}
	\begin{enumerate}[label=\arabic*)]
		\item \textbf{Tiempo de ejecución de CPU:} Tiempo real que la CPU emplea en computar una tarea específica.
		\item \textbf{Tiempo CPU del usuario:} Tiempo de CPU empleado en el propio programa.
		\item \textbf{Tiempo CPU del sistema:} Tiempo que la CPU emplea en realizar tareas el sistema operativo.
	\end{enumerate}
	
	\subsection*{Prestaciones de la CPU y sus Factores}
	\textbf{Relación de los ciclos de reloj, tiempo del ciclo de reloj y tiempo de la CPU:}
	\begin{itemize}[noitemsep]
		\item Frecuencia del reloj es inversa al tiempo del ciclo:
		\[ \text{Frecuencia del Reloj} = \frac{1}{\text{Tiempo del Ciclo}} \]
		\[ \text{Tiempo del Ciclo} = \frac{1}{\text{Frecuencia del Reloj}} \]
	\end{itemize}
	
	\subsection*{Prestaciones de las Instrucciones}
	\textbf{Número de ciclos de reloj requeridos por un programa:}
	\[ \text{Ciclos de Reloj} = \text{Número de Instrucciones} \times \text{CPI} \]
	
	\subsection*{Conceptos Claves (III)}
	\begin{enumerate}[label=\arabic*)]
		\item \textbf{Ciclos de reloj por instrucción (CPI):} Número medio de ciclos de reloj por instrucción para un programa o fragmento del programa.
	\end{enumerate}
	
	\subsection*{Ecuación Clásica de las Prestaciones de la CPU}
	Ahora se puede escribir la ecuación en términos del número de instrucciones, del CPI y del tiempo de ciclo:
	\[ \text{Tiempo\_CPU} = \text{Número de Instrucciones} \times \text{CPI} \times \text{Tiempo del Ciclo} \]
	O dada la frecuencia:
	\[ \text{Tiempo\_CPU} = \frac{\text{Número de Instrucciones} \times \text{CPI}}{\text{Frecuencia del Reloj}} \]
	
	\subsection*{Tabla de Conversión de Unidades}
	\begin{longtable}{ >{\RaggedRight}p{1.8cm} >{\centering}p{1cm} >{\centering}p{3.5cm} >{\RaggedRight}p{4cm} >{\RaggedRight\arraybackslash}p{4cm} }
		\toprule
		\textbf{Prefijo} & \textbf{Símbolo} & \textbf{Factor de Multiplicación} & \textbf{Equivalencia en Segundos (s)} & \textbf{Equivalencia en Hertz (Hz)} \\
		\midrule
		\endfirsthead % Encabezado para la primera página
		\multicolumn{5}{c}{\tablename~\thetable{}: Continuación} \\
		\toprule
		\textbf{Prefijo} & \textbf{Símbolo} & \textbf{Factor de Multiplicación} & \textbf{Equivalencia en Segundos (s)} & \textbf{Equivalencia en Hertz (Hz)} \\
		\midrule
		\endhead 
		\bottomrule
		\endfoot 
		\bottomrule
		\endlastfoot 
		
		Pico & p & $10^{-12}$ & 1 ps = 0.000000000001 s & 1 pHz = $10^{-12}$ Hz \\
		\addlinespace
		Nano & n & $10^{-9}$ & 1 ns = 0.000000001 s & 1 nHz = $10^{-9}$ Hz \\
		\addlinespace
		Micro & $\mu$ & $10^{-6}$ & 1 $\mu$s = 0.000001 s & 1 $\mu$Hz = $10^{-6}$ Hz \\
		\addlinespace
		Mili & m & $10^{-3}$ & 1 ms = 0.001 s & 1 mHz = $10^{-3}$ Hz \\
		\addlinespace
		(Unidad Base) & - & $10^{0}$ & 1 s & 1 Hz \\
		\addlinespace
		Kilo & k & $10^{3}$ & - & 1 kHz = 1,000 Hz \\
		\addlinespace
		Mega & M & $10^{6}$ & - & 1 MHz = 1,000,000 Hz \\
		\addlinespace
		Giga & G & $10^{9}$ & - & 1 GHz = 1,000,000,000 Hz \\
		
	\end{longtable}
	
\end{document}